%Festlegungen zum Dokument
\documentclass[a4paper,12pt]{article}

%Laden von verschiedenen Zusatzpaketen
%------------------------------------%

%Deutsches Sprachpaket
\usepackage[ngerman]{babel}

%Codierung dieser Datei ist UTF8
\usepackage[utf8]{inputenc}

%Pakete zum hinzufügen von Grafiken
\usepackage{graphicx}
\usepackage{graphics}

%Hinzufügen von Links (extern - INet) und Verlinkungen im Dokument (z.B. Überschriften)
%pdfborder entfernt Kästen um Links
\usepackage[pdfborder={0 0 0}]{hyperref}

%Ermöglicht das Ändern der Größe der Seitenränder
\usepackage{geometry}

%Ermöglicht modifizierte Kopf- und Fußzeile
\usepackage{fancyhdr}

%Allgemeine Einstellungen für das Dokument treffen
%------------------------------------%

%Größe der Seitenränder einstellen (benötigt Package: geometry)
\geometry{a4paper, left=30mm, right=20mm,top=25mm,bottom=25mm}

%Kopfzeile festlegen (benötigt Package: fancyhdr)
	%Bisherige Kopfzeile löschen
	\fancyhead{}
	%Kopfzeile links
	\fancyhead[L]{Ein Text links}
	%Kopfzeile rechts
	\fancyhead[R]{Ein Text rechts}
	%Statt R (Rechts) und L (Links) können auch weitere Positionen angegeben werden.
	%z.B. RO (Right Odd - Rechts Ungerade), RE (Right Even - Rechts Gerade), 
	%LO (Left Odd - Links Ungerade), LE (Left Even - Links Gerade)
	%Bei 2-seitigem Layout ist die kombinierte Angabe nützlich
	%z.B. \fancyhead[LO,RE]{...} (Bei ungerader Seite Links, bei gerader Seite rechts)

%Fußzeile festlegen (benötigt Package: fancyhdr)
	%Bisherige Fußzeile löschen
	\fancyfoot{}
	%Fußzeile rechts
	\fancyfoot[R]{\thepage}
	%\thepage - Seitenzahl

%Einrückung am Absatzabfang verhindern
\setlength{\parindent}{0em}


%Hier beginnt das eigentliche Dokument
%------------------------------------%
\begin{document}

%Nutze das Seitenlayout "empty" (ohne Kopf- und Fußzeile)
\pagestyle{empty}

%Inhaltsverzeichniss erzeugen
\tableofcontents

%Seitenumbruch
\newpage

%Nutze das Seitenlayout "fancy" mit angegebener Kopf- und Fußzeile (benötigt Package: fancyhdr)
\pagestyle{fancy}

%Erzeuge große Überschrift, welche eine Nummer erhält und im Inhaltsverzeichniss auftaucht
\section{Große Überschrift}
%Erzeuge mittlere Überschrift, welche eine Nummer erhält und im Inhaltsverzeichniss auftaucht
\subsection{Mittlere Überschrift}
%Erzeuge kleine Überschrift, welche eine Nummer erhält und im Inhaltsverzeichniss auftaucht
\subsubsection{Kleine Überschrift}

%Fülltext
Lorem ipsum dolor sit amet, consetetur sadipscing elitr, sed diam nonumy eirmod tempor invidunt ut labore et dolore magna aliquyam erat, sed diam voluptua. At vero eos et accusam et justo duo dolores et ea rebum. Stet clita kasd gubergren, no sea takimata sanctus est Lorem ipsum dolor sit amet. Lorem ipsum dolor sit amet, consetetur sadipscing elitr, sed diam nonumy eirmod tempor invidunt ut labore et dolore magna aliquyam erat, sed diam voluptua. At vero eos et accusam et justo duo dolores et ea rebum. Stet clita kasd gubergren, no sea takimata sanctus est Lorem ipsum dolor sit amet.

\newpage

\section{Aufzählung \& Nummerierung}
\subsection{Aufzählung}
%Erzeuge eine Aufzählung
\begin{itemize}
	%Erzeuge einen Stichpunkt
	\item Erster Punkt
	\item Zweiter Punkt
	\item Dritter Punkt
	%Erzeuge erste Schachtelungsebene
	\begin{itemize}
		\item Erster Punkt
		\item Zweiter Punkt
		\item Dritter Punkt
		%Erzeuge zweite Schachtelungsebene
		\begin{itemize}
			\item Erster Punkt
			\item Zweiter Punkt
			\item Dritter Punkt
			%Erzeuge dritte Schachtelungsebene (Maximum)
			\begin{itemize}
				\item Erster Punkt
				\item Zweiter Punkt
				\item Dritter Punkt
			\end{itemize}
		\end{itemize}
	\end{itemize}
\end{itemize}

\subsection{Nummerierung}
%Erzeuge eine Nummerierung
\begin{enumerate}
	%Erzeuge einen Stichpunkt
	\item Erster Punkt
	\item Zweiter Punkt
	\item Dritter Punkt
	%Erzeuge erste Schachtelungsebene
	\begin{enumerate}
		\item Erster Punkt
		\item Zweiter Punkt
		\item Dritter Punkt
		%Erzeuge zweite Schachtelungsebene
		\begin{enumerate}
			\item Erster Punkt
			\item Zweiter Punkt
			\item Dritter Punkt
			%Erzeuge dritte Schachtelungsebene (Maximum)
			\begin{enumerate}
				\item Erster Punkt
				\item Zweiter Punkt
				\item Dritter Punkt
			\end{enumerate}
		\end{enumerate}
	\end{enumerate}
\end{enumerate}

\subsection{Beliebiges Aufzählungszeichen}
\begin{itemize}
	\item [$\forall$] Beliebiges Aufzählungszeichen
	\item [$\rightarrow$] Beliebiges Aufzählungszeichen
	\item [$\Rightarrow$] Beliebiges Aufzählungszeichen
\end{itemize}

%Seitenumbruch
\newpage


%Erzeuge große Überschrift, ohne Nummer und ohne Eintrag im Inhaltsverzeichniss
\section*{Große Überschrift ohne Nummer}
%Erzeuge mittlere Überschrift, ohne Nummer und ohne Eintrag im Inhaltsverzeichniss
\subsection*{Mittlere Überschrift ohne Nummer}
%Erzeuge kleine Überschrift, ohne Nummer und ohne Eintrag im Inhaltsverzeichniss
\subsubsection*{Kleine Überschrift ohne Nummer}

%Internetlink (benötigt Package: hyperref)
%Sonderzeichen wie _ oder & müssen als \_ und \& geschrieben werden
Link: \url{http://www.loremipsum.de/}

%Einfügen eines Bildes (benötigt Packages: graphicx, graphics)
%Syntax: \includegraphics[width=Bildgröße]{Pfad zum Bild}
%Der Pfad muss relativ zur *.tex Datei sein
%Zur Angabe der Bildgröße empfiehlt sich die Relation zur Textweite (erhlät man mit \textwidth)
%\includegraphics[width=1\textwidth]{Bilder/pc.png}

\newpage

\section{Textformatierung}
\subsection{Ausrichtungen}
	\subsubsection{Zentriert}
		\begin{center}
			Lorem ipsum dolor sit amet, consetetur sadipscing elitr, sed diam nonumy eirmod tempor invidunt ut labore et dolore magna
			aliquyam erat, sed diam voluptua. At vero eos et accusam et justo duo dolores et ea rebum. Stet clita kasd gubergren, 
			no sea takimata sanctus est Lorem ipsum dolor sit amet. Lorem ipsum dolor sit amet, consetetur sadipscing elitr, 
			sed diam nonumy eirmod tempor invidunt ut labore et dolore magna aliquyam erat, sed diam voluptua. 
			At vero eos et accusam et justo duo dolores et ea rebum. Stet clita kasd gubergren, no sea takimata sanctus est 
			Lorem ipsum dolor sit amet.
		\end{center}
	\subsubsection{Links}
		\begin{flushleft}
			Lorem ipsum dolor sit amet, consetetur sadipscing elitr, sed diam nonumy eirmod tempor invidunt ut labore et dolore magna
			aliquyam erat, sed diam voluptua. At vero eos et accusam et justo duo dolores et ea rebum. Stet clita kasd gubergren, 
			no sea takimata sanctus est Lorem ipsum dolor sit amet. Lorem ipsum dolor sit amet, consetetur sadipscing elitr, 
			sed diam nonumy eirmod tempor invidunt ut labore et dolore magna aliquyam erat, sed diam voluptua. 
			At vero eos et accusam et justo duo dolores et ea rebum. Stet clita kasd gubergren, no sea takimata sanctus est 
			Lorem ipsum dolor sit amet.
		\end{flushleft}
	\subsubsection{Rechts}
		\begin{flushright}
			Lorem ipsum dolor sit amet, consetetur sadipscing elitr, sed diam nonumy eirmod tempor invidunt ut labore et dolore magna
			aliquyam erat, sed diam voluptua. At vero eos et accusam et justo duo dolores et ea rebum. Stet clita kasd gubergren, 
			no sea takimata sanctus est Lorem ipsum dolor sit amet. Lorem ipsum dolor sit amet, consetetur sadipscing elitr, 
			sed diam nonumy eirmod tempor invidunt ut labore et dolore magna aliquyam erat, sed diam voluptua. 
			At vero eos et accusam et justo duo dolores et ea rebum. Stet clita kasd gubergren, no sea takimata sanctus est 
			Lorem ipsum dolor sit amet.		
		\end{flushright}

\newpage
\subsection{Zeilenumbruch}
	%\\ bezeichnet einen Zeilenumbruch (alternativ \newline)
	Hier findet ein Zeilenumbruch \\ statt.
\subsection{Absatz}
	%\\ \\ dementsprechend einen Absatz
	Hier steht ein \\ \\ Absatz.
\subsection{Textformatierungen}
	%Schriebt einen Text Fett
	\textbf{Ich bin ein fettgedruckter Text} \\ 
	\textit{Ich bin ein kursiver Text}\\
	\textsl{Ich bin ein schräger Text}\\
	\emph{Ich bin ein hervorgehobener Text}\\
	\texttt{Ich bin in Schreibmaschine geschrieben}\\
	\textsc{Ich bin in Kapitälchen geschrieben}\\



%Hier endet das eigentliche Dokument
\end{document}